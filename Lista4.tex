\documentclass[a4paper, 12pt]{article}

\usepackage[portuges]{babel}
\usepackage[utf8]{inputenc}
\usepackage{amsmath}
\usepackage{indentfirst}
\usepackage{blindtext}
\usepackage{graphicx}
\usepackage[hidelinks]{hyperref}
\usepackage{gensymb}
\usepackage{pgfplots}

\author{Igor Abreu da Silva}

\title{Lista IV}

\begin{document}

    \begin{titlepage}
        \begin{center}
            \huge{Universidade Federal do Rio de Janeiro}
            \vspace{95pt}

            \large{Lista IV - Sistemas Lineares I}
            \vspace{160pt}
        \end{center}

        \begin{flushleft}
            \begin{tabbing}
                Alunos\qquad\qquad\= Igor Abreu da Silva\\
                DRE\> 112053874 \\
                Curso\> Engenharia Eletrônica \\
                Turma\> 2016/2 \\
                Professor\> Natanael Nunes de Moura Junior \\

            \end{tabbing}

        \end{flushleft}

        \begin{center}
            \vspace{\fill}
            Rio de Janeiro, 22 de Novembro de 2016
        \end{center}
    \end{titlepage}

    \newpage
    \tableofcontents
    \listoffigures
    \thispagestyle{empty}
    \newpage
    \pagenumbering{arabic}

	\section{Série de Fourier}
		\subsection{Questão 1}    
			\subsubsection{Item a}
			\[\frac{2\pi}{3}; \frac{5\pi}{3} \rightarrow MMC(6\pi; 15\pi) \Rightarrow \omega_{0} = 30\pi \Rightarrow T_{0} = \frac{2\pi}{\omega _{0}} = \frac{1}{15}\]
			\subsubsection{Item b}
			\[\sum_{n=-\infty}^{+\infty} a_{n}e^{j\omega _{0}t} \Rightarrow a_{n} = \frac{1}{\frac{1}{15}} \]
			\subsubsection{Item c}						
		\subsection{Questão 2}    
		\subsection{Questão 3} 
			\subsubsection{Item a}
			\subsubsection{Item b}			   
		\subsection{Questão 4}    
			\subsubsection{Item a}
			\subsubsection{Item b}			   		
		\subsection{Questão 5}    
			\subsubsection{Item a}
			\subsubsection{Item b}			   		
		\subsection{Questão 6}    				
			\subsubsection{Item a}
			\subsubsection{Item b}			   								
	\section{Transformada de Fourier}    
		\subsection{Questão 7} 
			\subsubsection{Item a}
			\subsubsection{Item b}			   		   
			\subsubsection{Item c}			   			
		\subsection{Questão 8}    
		\subsection{Questão 9}  
			\subsubsection{Item a}
			\subsubsection{Item b}			  
		\subsection{Questão 10}    
		\subsection{Questão 11}     
			\subsubsection{Item a}
			\subsubsection{Item b}			  		
\end{document}